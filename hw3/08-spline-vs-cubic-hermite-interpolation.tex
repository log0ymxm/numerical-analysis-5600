\textbf{Spline versus Cubic Hermite Interpolation.} Let the function
$s(x)$ be defined by
\[
s(x) = \left\{
\begin{aligned}
(\gamma - 1)(x^3 - x^2) + x + 1 &\,\,\,\, \text{if}\,\, x\in [0,1] \\
\gamma x^3 - 5 \gamma x^2 + 8 \gamma x - 4 \gamma + 2 &\,\,\,\, \text{if}\,\, x\in [1,2]
\end{aligned}
\right.
\]

\begin{enumerate}
\item Show that $s$ is piecewise cubic Hermite interpolant to the
  data:
\[
s(0)=1, \;\;\; s(1)=s(2)=2, \;\;\; s^\prime(0)=1, \;\;\; s^\prime(1) =
\gamma, \;\;\; s^\prime(2)=0
\]
\item For what value of $\gamma$ does $s$ become a cubic spline?
\end{enumerate}

{\color{blue}

\[
\begin{aligned}
s^\prime(x) = \left\{
\begin{aligned}
(\gamma - 1)(3 x^2 - 2x) + 1 &\,\,\,\, \text{if}\,\, x\in [0,1] \\
3 \gamma x^2 - 10 \gamma x + 8 \gamma &\,\,\,\, \text{if}\,\, x\in [1,2]
\end{aligned}
\right.
\end{aligned}
\]

\begin{enumerate}
\item
\begin{align*}
s(0) &= (\gamma - 1)(3(0) - 2(0)) + 0 + 1 = 1 \\
s(1) &= (\gamma - 1)(3(1) - 2(1)) + 0 + 1 = 2 \\
s(1) &= (\gamma - 1)(3(1) - 2(1)) + 0 + 1 = 2 \\
     &= \gamma (1)^3 - 5 \gamma (1)^2 + 8 \gamma (1) - 4 \gamma + 2\\
     &= 4 \gamma + 8 \gamma - 4 \gamma + 2 \\
     &= 2 \\
s(2) &= \gamma (2)^3 - 5 \gamma (2)^2 + 8 \gamma (2) - 4 \gamma + 2\\
     &= 8 \gamma - 20 \gamma + 16 \gamma - 4\gamma + 2\\
     &= 2 \\
s^\prime(0) &= (\gamma - 1)(3 (0)^3 - 2(0)) + 1 \\
           &= 1 \\
s^\prime(1) &= (\gamma - 1)(3 (1)^2 - 2(1)) + 1 \\
           &= \gamma - 1 + 1 \\
           &= 1 \\
           &= 3 \gamma (1)^2 - 10 \gamma (1) + 8 \gamma \\
           &= \gamma \\
s^\prime(2) &= 3 \gamma (2)^2 - 10 \gamma (2) + 8 \gamma \\
           &= 12 \gamma - 20 \gamma + 8 \gamma \\
           &= 0 \\
\end{align*}

Therefore $s$ interpalates to the data.

\item

For $s(x)$ to be a cubic spline it must have equivalent values at the
interpolated points for the second derivative. We check the values for
$s^{\prime\prime}(x)$ at 1. Here we find that the only possible value
for $\gamma$ is $1/2$.

\begin{align*}
% s''(x)
s^{\prime\prime}(x) = \left\{
\begin{aligned}
(\gamma - 1)(6x - 2)
&\,\,\,\, \text{if}\,\, x\in [0,1] \\
6 \gamma x - 10 \gamma
&\,\,\,\, \text{if}\,\, x\in [1,2]
\end{aligned}
\right.
\end{align*}


\begin{align*}
s^{\prime\prime}(x) &= (\gamma - 1) (6x - 2) \\
                 &= 6 \gamma x - 6 x - 2 \gamma + 2 \\
                 &= 6 (1/2) x - 6 x - 2 (1/2) + 2\\
                 &= - 3 x + 1 = -2 \,\,\,\text{when } x=1 \\
s^{\prime\prime}(x) &= 6 \gamma x - 10 \gamma \\
                 &= - 4 \gamma = - 4 (1/2) = -2\\
     \Rightarrow \gamma &= \frac{1}{2}
\end{align*}


\end{enumerate}

}
