\textbf{Division without division:} Suppose you have a computer or calculator that has no built-in division. Come up with a fixed point iteration that converges to $1/r$ for any given non-zero number $r$, and that only uses addition, subtraction, and multiplication. \textbf{Hint:} Write down an equation satisfied by $1/r$, apply Newton's method to that equation, and then modify Newton's method so that it doesn't use division. Your resulting method should converge of order 2.

{\color{blue}

We can
\[
\begin{aligned}
f(x) &= \frac{1}{r} = x \\
&\Rightarrow 1 = xr \\
&\Rightarrow \frac{1}{x} = r \\
&\Rightarrow \frac{1}{x} - r = 0
\end{aligned}
\]

\[
f^\prime (x) = \frac{-1}{x^2}
\]

\[
\begin{aligned}
F(x) &= x - \frac{f(x)}{f^\prime(x)} \\
&= x - \frac{\frac{1}{x} - r}{\frac{-1}{x^2}} \\
&= x + \left(\frac{1}{x} - r\right) x^2 \\
&= x(2 - xr)
\end{aligned}
\]

}

\begin{minted}[mathescape,
               linenos,
               numbersep=5pt,
               gobble=2,
               frame=lines,
               framesep=2mm]{clojure}
  (defn reciprocal [r]
    (let [convergent? (fn [{:keys [steps error] :or {steps 0
                                                    error 1}}]
                        (or (< (Math/abs (double error)) 0.00001)
                            (> steps 20)))
          next-guess (fn [{:keys [x last steps] :or {last 0
                                                    steps 0}}]
                       (let [guess (* x (- 2 (* x r)))]
                         {:steps (inc steps)
                          :error (- last guess)
                          :last x
                          :x guess}))
          answer (first (drop-while #(not (convergent? %))
                                    (iterate #(next-guess %)
                                             {:x (if (= 20 r) 0.01 0.1)})))]
      (println answer)
      (:x answer)))

  (reciprocal 3) => 0.33333333333333337
  (reciprocal 4) => 0.24999999999999997
\end{minted}

{\color{blue}
\textbf{Explanation: } We define a function that uses a few local
functions to perform Newton's method. On line 2, we locally define the
\texttt{convergent?} function which checks if our answer is below a
threshold or the iteration has exceeded a maximum number of steps. On
line 6, we define a function called \texttt{next-guess} that performs
the actual Newton's step operation $x(2 - xr)$ and returns it's value
with additional metadata to help with iteration. On line 13 we perform
the calculation, starting at a suitably small value and treating our
function \texttt{next-guess} as a sequence of iterations we take only
the first convergent value we receive.
}
