\textbf{The Benstein B\'ezier Form.} With the notation given in our
handout, show that every univariate polynomial of degree $d$ can be
written uniquely in Bernstein-B\'ezier form.

{\color{blue}

For any univariate polynomial the interpolated polynomial for the
Bernstein B\'ezier form will be the unique polynomial of degree $d$,
and it will interpolate to the univariate polynomial.

{\color{blue}

For single variable let $b_1 = t$ and $b_2 = 1 - t$ (this is
equivalent to the B-Form on the interval $[0,1]$, and can be converted
to any other interval which will add more constants to equations, but
they will not have an effect on the final result, so we are keeping
them out of the way) then,

\begin{align*}
\sum_{i=0}^d {d \choose i} c_{i,d-i} t^i (1-t)^{d-i}
&= \sum_{i=0}^d {d \choose i} c_{i,d-i} t^i \sum_{k=0}^{d-i} (-1)^k {d-i \choose k} t^k \\
&= \sum_{i=0}^d c_{i,d-i} \sum_{k=0}^{d-i} (-1)^k {d \choose i} {d-i \choose k} t^{k+i} \\
&= \sum_{i=0}^d c_{i,d-i} \sum_{k=i}^d (-1)^{k-i} {d \choose i} {d-i \choose k-i} t^k \\
&= \sum_{i=0}^d \sum_{k=i}^d c_{i,d-i} (-1)^{k-i} {d \choose i} {d \choose k} t^k \\
\end{align*}

which is a power form.

}

}
