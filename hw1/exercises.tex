\documentclass[11pt, oneside]{article}   	% use "amsart" instead of "article" for AMSLaTeX format
\usepackage{geometry}                		% See geometry.pdf to learn the layout options. There are lots.
\geometry{letterpaper}                   		% ... or a4paper or a5paper or ... 
%\geometry{landscape}                		% Activate for for rotated page geometry
%\usepackage[parfill]{parskip}    		% Activate to begin paragraphs with an empty line rather than an indent
\usepackage{graphicx}				% Use pdf, png, jpg, or eps� with pdflatex; use eps in DVI mode
								% TeX will automatically convert eps --> pdf in pdflatex		
\usepackage{amssymb}
\usepackage{amsmath}

\usepackage{minted}
\usepackage{hyperref}

\usepackage{color}

\newcommand{\blue}[1]{\textcolor{blue}{#1}}
\newcommand{\pe}[1]{\blue{PE: #1}}

\newenvironment{polynomial}
  {\par\vspace{\abovedisplayskip}%
   \setlength{\leftskip}{\parindent}%
   \setlength{\rightskip}{\leftskip}%
   \medmuskip=4mu plus 2mu minus 2mu
   \binoppenalty=0
   \noindent$\displaystyle}
  {$\par\vspace{\belowdisplayskip}}



\newcommand{\ans}[1]{\blue{
\begin{polynomial}
#1
\end{polynomial}
}}


\newcommand{\vect}[1]{\vec{\lowercase{#1}}}

\title{Homework \#1}
\author{
Haylee Crane \\
Paul English \\
Sean Melessa \\
Sarath Thekkedath
}
%\date{}							% Activate to display a given date or no date

\begin{document}
\maketitle
%\section{}
%\subsection{}

\begin{enumerate}
\item[Exercise 1.] Find a formula that describes the trajectory of the point $\mathbf{O}$ in cartesian coordinates as a function of time.

\blue{
\begin{polynomial}
\left(
   \begin{matrix} % or pmatrix or bmatrix or Bmatrix or ...
      cos(\frac{2\pi t}{s}) & -sin(\frac{2 \pi t}{s}) & 0 \\
      sin(\frac{2 \pi t}{s}) & cos(\frac{2 \pi t}{s}) & 0 \\
      0 & 0 & 1 \\
   \end{matrix}\right)
\left(
\begin{matrix}
x_0 \\
y_0 \\
z_0
\end{matrix}
\right)
=
\left(
\begin{matrix}
x_t \\
y_t \\
z_t
\end{matrix}
\right)
\end{polynomial}
}

\blue{
where $s$ is the period of a sidereal day, $\mathbf{u}$ is the latitude of our coordinate, and $\mathbf{v}$ is the longitude of our coordinate.
}

\item[Exercise 2.] Write a program that converts angles from degrees, minutes, and seconds to radians and vice versa. Make sure your program does what it's supposed to do.

\begin{minted}[mathescape,
               linenos,
               numbersep=5pt,
               gobble=2,
               frame=lines,
               framesep=2mm]{clojure}
  
    (s/defn dms->radians :- RadCoordinateList
      [A :- DMSCoordinateList]
      (with-precision 20
        (let [->rad (/ @pi 180)
              deg->rad (* 1 ->rad)
              m->rad (* 1/60 ->rad)
              s->rad (* 1/3600 ->rad)
              times (mmul A (transpose [[1 0 0 0 0 0 0 0 0 0]]))
              orientations (mmul A (transpose [[0 0 0 0 1 0 0 0 0 0]
                                               [0 0 0 0 0 0 0 0 1 0]]))
              heights (mmul A (transpose [[0 0 0 0 0 0 0 0 0 1]]))
              radians (->> [[0 deg->rad m->rad s->rad 0 0 0 0 0 0]
                            [0 0 0 0 0 deg->rad m->rad s->rad 0 0]]
                           transpose
                           (mmul A)
                           (* orientations))
              ]
          (parse-rad-list
           (join-1 times radians heights)))))

    (s/defn radians->dms :- DMSCoordinateList
      [A :- RadCoordinateList]
      (let [times (mmul A (transpose [[1 0 0 0]]))
            heights (mmul A (transpose [[0 0 0 1]]))
            degrees-decimal (->> [[0 (/ 180 @pi) 0 0]
                                  [0 0 (/ 180 @pi) 0]]
                                 (with-precision 20)
                                 transpose
                                 (mmul A))
            orientations (emap #(if (pos? %) 1 -1) degrees-decimal)
            positive-degrees (emap abs degrees-decimal)
            degrees (emap #(Math/floor %) positive-degrees)
            minutes-decimal (* 60 (- positive-degrees degrees))
            minutes (emap #(Math/floor %) minutes-decimal)
            seconds (* 60 (- minutes-decimal minutes))]
        (parse-dms-list
         (join-1 times
                 (join-1-interleave degrees minutes seconds orientations)
                 heights))))
\end{minted}

\blue{
\textbf{Explanation:} On line 1 and 2 we define a function that converts a degrees, minutes, seconds representation to radians. It outputs a list of the form $t \psi \lambda h$ defined as a $RadCoordinateList$, and accepts one value $A$ a matrix consisting of coordinates in the form given at (8). In line 3 we ensure our function uses $BigDecimal$ precision of 20 places. Lines 4-7 calculate the scalar values that are needed to convert to radians. Then in lines 8-11 we separate $t$, $NS/EW$ orientations, and $h$ using matrix operations. Finally starting on line 12 we use a matrix operation with the radian constants and our orientations to perform the actual $dms\to radians$ computation (The $->>$ operator means thread an argument through the last parameter, it just shortens the code). Lastly on lines 18 and 19 we join $t$ and $h$ onto our calculated radians on dimension 1, and ensure that all numbers are parsed to match our output schema.}

\blue{Full source here: \url{https://github.com/log0ymxm/gps-sim/blob/master/src/gps_sim/utils/angles.clj}}

\blue{Test cases available here: \url{https://github.com/log0ymxm/gps-sim/blob/master/test/gps_sim/utils/angles_test.clj}}

\item[Exercise 3.] Find a formula that converts position as given in (8) at time $t = 0$ into cartesian coordinates.

\blue{
With the data schema presented by (8) we have access to latitude $\psi$ and longitude $\lambda$. Assuming we've already converted these to radians, we can calculate the cartesian coordinates by transforming from spherical coordinates as follows. Note traditionally with spherical coordinates we measure $\phi$ as the angle coming off the $z$-axis, with latitude $\psi$ we measure the angle coming from the equator or $xy$-plane, thus $\psi = \pi/2 - \phi$, and we use $cos$ in calculating $x,y$ and $sin$ to calculate $z$.
\begin{polynomial}
\begin{aligned}
x &= \rho \, cos(\psi) cos(\lambda) \\ 
y &= \rho \, cos(\psi) sin(\lambda) \\
z &= \rho \, sin(\psi)
\end{aligned}
\end{polynomial}
}

\blue{where $\rho = (R + h)$.}

\item[Exercise 4.] Find a formula that converts position and general time $t$ as given in (8) into cartesian coordinates.
\blue{
\begin{polynomial}
\begin{aligned}
x &= \rho \, cos\left(\psi \right) cos \left(\frac{2 \pi t}{s} + \lambda \right) \\ 
y &= \rho \, cos\left(\psi \right) sin \left(\frac{2 \pi t}{s} + \lambda \right) \\
z &= \rho \, sin\left(\psi \right)
\end{aligned}
\end{polynomial}
}

\item[Exercise 5.] Find a formula that converts a position given in cartesian coordinates at time $t = 0$ into a position of the form (8).
\blue{
\begin{polynomial}
\begin{aligned}
\rho &= \sqrt{x^2 + y^2 + z^2} \\ 
\psi &= asin(z / \rho)\\
\lambda &= atan (y / x) \\
\end{aligned}
\end{polynomial}
}

\blue{
Here we convert from cartesian coordinates into spherical longitude and latitude. You would then use a routine to convert from radians to degrees, minutes, seconds to get a position of form (8). The function in answer 2 would do this.
}

\item[Exercise 6.] Find a formula that converts general time $t$ and a position given in cartesian coordinates into a position of the form (8).
\blue{
\begin{polynomial}
\begin{aligned}
\rho &= \sqrt{x^2 + y^2 + z^2} \\ 
\psi &= asin(z / \rho)\\
\lambda &= atan (y / x) - \frac{2 \pi t}{s} \\
\end{aligned}
\end{polynomial}
}
\blue{
We use the same formulas from answer 5, but to account for time we adjust the $\lambda$ term using $\frac{2 \pi t}{s}$.
}

\item[Exercise 7.] Find a formula that describes the trajectory of lamp post B12 in cartesian coordinates as a function of time.

\blue{
\begin{polynomial}
\begin{aligned}
\psi &= (1)\frac{\pi}{180} \left( 40 + 45/60 + 55.0/3600  \right) = 0.7114883177\\
\lambda &= (-1)\frac{\pi}{180} \left( 111 + 50/60 + 58.0/3600  \right) = -1.952141072\\
\rho &= (R + 1372.0) \\
x &= \rho \, cos(\psi) cos(\frac{2 \pi t}{s} + \lambda) \\ 
y &= \rho \, cos(\psi) sin(\frac{2 \pi t}{s} + \lambda) \\
z &= \rho \, sin(\psi)
\end{aligned}
\end{polynomial}
}

\blue{First we calculate $\psi$ and $\lambda$ in radians, then $\rho$ is calculated. We use these values with the last 3 equations to calculate the cartesian coordinates of our point B12 given some time $t$.}

\item[Exercise 8.] Given a point $\vect{x}$ on earth and a point $\vect{s}$ in space, both in cartesian coordinates, find a condition that tells you whether $\vect{s}$ as viewed from $\vect{x}$ is above the horizon.

\blue{Since the horizon line is going to be the orthogonal vector of a vehicle $\vect{v}$, we can project a satellite $\vect{s}$ onto our vehicle and determine where it's oriented around our vehicle. If the projected value is greater than the magnitude $||\vect{v}||$ we know it's above the horizon. This also let's us simplify our inequality to the second equation below.}

\blue{
\begin{polynomial}
\begin{aligned}
\frac{\vect{v} \cdot \vect{s}}{||\vect{v}||} &> ||\vect{v}|| \\
\vect{v} \cdot \vect{s} &> ||\vect{v}||^2
\end{aligned}
\end{polynomial}
}

\item[Exercise 9.] Discuss how to compute $t_s$ and $\vect{x}_s$.
\blue{
\begin{polynomial}
\begin{aligned}
\vect{x}_s(t) &= (R + h) \left[ \vect{u} cos\left(\frac{2 \pi t}{p} + \theta \right) + \vect{v} sin \left(\frac{2 \pi t}{p} + \theta \right) \right] \\
F(t_s) &= t_V - \frac{|| \vect{x}_s(t_s) - \vect{x}_v ||}{c} - t_s = 0 \\
t_s &= t_V - \frac{|| \vect{x}_s(t_s) - \vect{x}_v||}{c}
\end{aligned}
\end{polynomial}
}

\blue{
The function to solve for $x_s$ is a function of time, and it's required to solve for time seen in the last line above. Thus we use an iterative method like Newton's method or a fixed-point iteration to approach an approximate answer given a logical starting point.
}

\item[Exercise 10.] Suppose you have data of the form from (11) from 4 satellites. Write down a set of four equations whose solutions are the position of the vehicle in cartesian coordinates, and $t_V$

\blue{
\begin{polynomial}
\begin{aligned}
|| \vect{x}_V - \vect{x}_{s_1} || - || \vect{x}_V - \vect{x}_{s_2} || + c(t_{s_2} - t_{s_1}) &= 0 \\
|| \vect{x}_V - \vect{x}_{s_1} || - || \vect{x}_V - \vect{x}_{s_3} || + c(t_{s_3} - t_{s_1}) &= 0 \\
|| \vect{x}_V - \vect{x}_{s_1} || - || \vect{x}_V - \vect{x}_{s_4} || + c(t_{s_4} - t_{s_1}) &= 0 \\
\end{aligned}
\end{polynomial}
\begin{polynomial}
\begin{aligned}
t_V = t_s + \frac{|| \vect{x}_s(t_s) - \vect{x}_v||}{c}
\end{aligned}
\end{polynomial}
}
\blue{
The first 3 equations form the system of equations, that can be solved to find the position of the vehicle $\vect{x}_V$. They are derived using equation (12), $||\vect{x}_V - \vect{x}_s|| = c(t_V - t_s)$. We end up with 4 equations of the form (12), one for each satellite. Since $t_V$ is an unknown variable we use the 4 equations of form (12), and using subtraction derive the three above, eliminating the unknown $t_V$. This solves for $\vect{x}_V$. Once we know $\vect{x}_V$ we're able to use the last equation above to solve for $t_V$.}

\item[Exercise 11.] Suppose you have data of the form (11) from more than 4 satellites. Write down a least squares problem whose solution the position of the vehicle in cartesian coordinates, and $t_V$.

\blue{
A least squares problem solves a linear system of the form $A\vect{x} = \vect{b}$ where $\vect{x}$ is unknown. Algebraically we could solve for do this using $\vect{x} = A^{-1} \vect{b}$ however not all matrices have an inverse, and even if they do it can be expensive to compute them. Because of this we choose a value of $\vect{x}$ that minimizes the least-squares form $||A\vect{x} - \vect{b}||^2$.
}
\blue{
\begin{polynomial}
\begin{aligned}
A\vect{x} &=\vect{b} \\
A\vect{x} - \vect{b} &= 0 \\
min(||Ax-\vect{b}||^2) &\approx 0
\end{aligned}
\end{polynomial}
}

\blue{With 4 satellites we have three equations $F_m(x_V)$, we put these into the array $F(\vect{x}_V)$. We can then solve a least squares problem when we find the dot product of the vector $F(\vect{x}_V)$. This is seen in $f$.
}
\blue{
\begin{polynomial}
\begin{aligned}
F_m(x_V) &= ||\vect{x}_V - \vect{x}_{s_1}|| - ||\vect{x}_V - \vect{x}_{s_{m+1}} || + c (t_{s_{m+1}} - t_{s_1}) = 0 \\
F(\vect{x}_V) &= \left(\begin{matrix}
F_1(\vect{x}_V) \\
F_2(\vect{x}_V) \\
\vdots \\
F_m(\vect{x}_V)
\end{matrix}\right) \\
f &= F^\intercal F \\
&= F_1^2 + F_2^2 + \cdots + F_m^2 \\
&= || F ||^2 \\
\end{aligned}
\end{polynomial}
}

\item[Exercise 12.] Find a formula for the \textit{ground track} of satellite 1, i.e. the position in geographic coordinates directly underneath the satellite on the surface of the earth, as a function of time. Do you notice anything particular? What is the significance of the orbital period being exactly one half sidereal day?
\blue{
\begin{polynomial}
\begin{aligned}
\vect{x}_s(t) &= R \left[ \vect{u} cos(\frac{2 \pi t}{p} + \theta) + \vect{v} sin(\frac{2 \pi t}{p} + \theta) \right] \\
\end{aligned}
\end{polynomial}
If the orbital period of a satellite is half that of a sidereal day, it means it orbits twice for each single rotation of the earth. So this means that the satellite sees each point on the ground track exactly once in a sidereal day. 
}

\item[Exercise 13.] Find a precise description of Newton's method as it is applied to the nonlinear system obtained by processing data from 4 satellites, as derived in an earlier exercise. Your answer should include an explicit specification of the derivatives involved.

\blue{
\begin{polynomial}
\begin{aligned}
\vect{x}_s(t) &= (R + h) [ \vect{u} cos(\frac{2 \pi t}{p} + \theta) + \vect{v} sin(\frac{2 \pi t}{p} + \theta)] \\
\vect{x}_s^\prime(t) &= (R + h) \left[ \frac{2 \pi}{p} \left( -\vect{u} sin(\frac{2 \pi t}{p} + \theta) + \vect{v} cos(\frac{2 \pi t}{p} + \theta) \right) \right] \\
\end{aligned}
\end{polynomial}
}
\blue{
If $\vect{x}_V = \left(\begin{matrix} x_V \\ y_V \\ z_V \end{matrix}\right)$ 
and $\vect{x}_{s_j} = \left(\begin{matrix} x_{s_j}\\ y_{s_j} \\ z_{s_j} \end{matrix}\right)$ 
for $j = 1,2,3$,\\
$F(x_V) = \left(\begin{matrix} F_1 \\ F_2 \\ F_3 \end{matrix}\right) = \left(\begin{matrix}
|| \vect{x}_V - \vect{x}_{s_1} || - || \vect{x}_V - \vect{x}_{s_2} || - c(t_{s_2} - t_{s_1}) \\
|| \vect{x}_V - \vect{x}_{s_1} || - || \vect{x}_V - \vect{x}_{s_3} || - c(t_{s_3} - t_{s_1}) \\
|| \vect{x}_V - \vect{x}_{s_1} || - || \vect{x}_V - \vect{x}_{s_4} || - c(t_{s_4} - t_{s_1})
\end{matrix}\right) \rightarrow$ \\
\begin{polynomial}
\begin{aligned}
\nabla F = \left(\begin{matrix} \nabla F_1 \\ \nabla F_2 \\ \nabla F_3 \end{matrix}\right) &=
\left(\begin{matrix}
\frac{\partial}{\partial x_V} F_1 & \frac{\partial}{\partial y_V} F_1 & \frac{\partial}{\partial z_V} F_1 \\
\frac{\partial}{\partial x_V} F_2 & \frac{\partial}{\partial y_V} F_2 & \frac{\partial}{\partial z_V} F_2 \\
\frac{\partial}{\partial x_V} F_3 & \frac{\partial}{\partial y_V} F_3 & \frac{\partial}{\partial z_V} F_3
\end{matrix}
\right) \\
&= \left(\begin{matrix}
\frac{x_V - x_{s_1}}{|| \vect{x}_V - \vect{x}_{s_1} ||} - \frac{x_V - x_{s_2}}{|| \vect{x}_V - \vect{x}_{s_2} ||} &
\frac{y_V - y_{s_1}}{|| \vect{x}_V - \vect{x}_{s_1} ||} - \frac{y_V - y_{s_2}}{|| \vect{x}_V - \vect{x}_{s_2} ||} &
\frac{z_V - z_{s_1}}{|| \vect{x}_V - \vect{x}_{s_1} ||} - \frac{z_V - z_{s_2}}{|| \vect{x}_V - \vect{x}_{s_2} ||} \\
\frac{x_V - x_{s_1}}{|| \vect{x}_V - \vect{x}_{s_1} ||} - \frac{x_V - x_{s_3}}{|| \vect{x}_V - \vect{x}_{s_3} ||} &
\frac{y_V - y_{s_1}}{|| \vect{x}_V - \vect{x}_{s_1} ||} - \frac{y_V - y_{s_3}}{|| \vect{x}_V - \vect{x}_{s_3} ||} &
\frac{z_V - z_{s_1}}{|| \vect{x}_V - \vect{x}_{s_1} ||} - \frac{z_V - z_{s_3}}{|| \vect{x}_V - \vect{x}_{s_3} ||} \\
\frac{x_V - x_{s_1}}{|| \vect{x}_V - \vect{x}_{s_1} ||} - \frac{x_V - x_{s_4}}{|| \vect{x}_V - \vect{x}_{s_4} ||} &
\frac{y_V - y_{s_1}}{|| \vect{x}_V - \vect{x}_{s_1} ||} - \frac{y_V - y_{s_4}}{|| \vect{x}_V - \vect{x}_{s_4} ||} &
\frac{z_V - z_{s_1}}{|| \vect{x}_V - \vect{x}_{s_1} ||} - \frac{z_V - z_{s_4}}{|| \vect{x}_V - \vect{x}_{s_4} ||}
\end{matrix}\right).
\end{aligned}
\end{polynomial}
}
\blue{
\begin{polynomial}
\begin{aligned}
\text{Solve } \left\{
\begin{matrix}
\nabla F_{x_V} &= x_V^{(k)} S^{(k)} = -F(x_V^{(k)}),\\
x_V^{(k+1)} &= x_V^{(k)} + S^{(k)}.
\end{matrix}
\right.
\end{aligned}
\end{polynomial}
}

\item[Exercise 14.] Similarly, find Newton's method for the nonlinear system obtained from the least squares approach. Again, your answer should include an explicit specification of the derivatives involved.

\blue{
If $\vect{x}_V = \left(\begin{matrix} x_V \\ y_V \\ z_V \end{matrix}\right)$ and $\vect{x}_{s_j} = \left(\begin{matrix} x_{s_j}\\ y_{s_j} \\ z_{s_j} \end{matrix}\right)$ for $j = 1,2,\hdots,m$, \\
$F(x_V) = \left(\begin{matrix} F_1 \\ F_2 \\ \vdots \\ F_{m-1} \end{matrix}\right) = \left(\begin{matrix}
|| \vect{x}_V - \vect{x}_{s_1} || - || \vect{x}_V - \vect{x}_{s_2} || - c(t_{s_2} - t_{s_1}) \\
|| \vect{x}_V - \vect{x}_{s_1} || - || \vect{x}_V - \vect{x}_{s_3} || - c(t_{s_3} - t_{s_1}) \\
\vdots \\
|| \vect{x}_V - \vect{x}_{s_1} || - || \vect{x}_V - \vect{x}_{s_m} || - c(t_{s_m} - t_{s_1})
\end{matrix}\right) \rightarrow$ \\
\begin{polynomial}
\begin{aligned}
\nabla F = \left(\begin{matrix} \nabla F_1 \\ \nabla F_2 \\ \vdots \\ \nabla F_{m-1} \end{matrix}\right) &=
\left(\begin{matrix}
\frac{\partial}{\partial x_V} F_1 & \frac{\partial}{\partial y_V} F_1 & \frac{\partial}{\partial z_V} F_1 \\
\frac{\partial}{\partial x_V} F_2 & \frac{\partial}{\partial y_V} F_2 & \frac{\partial}{\partial z_V} F_2 \\
\vdots & \vdots & \vdots \\
\frac{\partial}{\partial x_V} F_{m-1} & \frac{\partial}{\partial y_V} F_{m-1} & \frac{\partial}{\partial z_V} F_{m-1}
\end{matrix}\right)
\end{aligned}
\end{polynomial}
}
\blue{
\begin{polynomial}
\begin{aligned}
&= \left(\begin{matrix}
\frac{x_V - x_{s_1}}{|| \vect{x}_V - \vect{x}_{s_1} ||} - \frac{x_V - x_{s_2}}{|| \vect{x}_V - \vect{x}_{s_2} ||} &
\frac{y_V - y_{s_1}}{|| \vect{x}_V - \vect{x}_{s_1} ||} - \frac{y_V - y_{s_2}}{|| \vect{x}_V - \vect{x}_{s_2} ||} &
\frac{z_V - z_{s_1}}{|| \vect{x}_V - \vect{x}_{s_1} ||} - \frac{z_V - z_{s_2}}{|| \vect{x}_V - \vect{x}_{s_2} ||} \\
\frac{x_V - x_{s_1}}{|| \vect{x}_V - \vect{x}_{s_1} ||} - \frac{x_V - x_{s_3}}{|| \vect{x}_V - \vect{x}_{s_3} ||} &
\frac{y_V - y_{s_1}}{|| \vect{x}_V - \vect{x}_{s_1} ||} - \frac{y_V - y_{s_3}}{|| \vect{x}_V - \vect{x}_{s_3} ||} &
\frac{z_V - z_{s_1}}{|| \vect{x}_V - \vect{x}_{s_1} ||} - \frac{z_V - z_{s_3}}{|| \vect{x}_V - \vect{x}_{s_3} ||} \\
\vdots & \vdots & \vdots \\
\frac{x_V - x_{s_1}}{|| \vect{x}_V - \vect{x}_{s_1} ||} - \frac{x_V - x_{s_{m-1}}}{|| \vect{x}_V - \vect{x}_{s_{m-1}} ||} &
\frac{y_V - y_{s_1}}{|| \vect{x}_V - \vect{x}_{s_1} ||} - \frac{y_V - y_{s_{m-1}}}{|| \vect{x}_V - \vect{x}_{s_{m-1}} ||} &
\frac{z_V - z_{s_1}}{|| \vect{x}_V - \vect{x}_{s_1} ||} - \frac{z_V - z_{s_{m-1}}}{|| \vect{x}_V - \vect{x}_{s_{m-1}} ||}
\end{matrix}\right).
\end{aligned}
\end{polynomial}
Since $\nabla F |_{x_V=x_V^{(k)}} S^{(k)} = -F(x_V^{(k)})$ is overdetermined and might not have an exact solution, use nonlinear least squares to find an approximation,
\begin{polynomial}
\begin{aligned}
G(S^{(k)}) &= \nabla F |_{x_V=x_V^{(k)}} S^{(k)} + F(x_V^{(k)}) = 0 \rightarrow \\
g(S^{(k)}) &= G(S^{(k)})^TG(S^{(k)}).
\end{aligned}
\end{polynomial}
We want to find $S^{(k,l)}$ such that the error $g(S^{(k,l)})$ is a minimum, so we use Newton's Method again on $\nabla g = 0$, where the gradient is taken with respect to the vector $S^{(k)},$
\begin{polynomial}
\begin{aligned}
\text{Solve } \left\{
    \begin{array}{lr}
	(\nabla^2 g) S'^{(l)} = -\nabla g|_{S^{(k)} = S^{(k,l)}},\\
	S^{(k,l+1)} = S^{(k,l)} + S'^{(l)}.
     \end{array}
   \right.
\end{aligned}
\end{polynomial}
}
\blue{
$\nabla g = \nabla\left(\sum\limits_{j=1}^{m-1} || \nabla F_j|_{x_V=x_V^{(k)}} \cdot S^{(k)} + F_j(x_V^{(k)}) ||^2 \right)$.\\
Since $\nabla^2 g = \nabla^2 G \cdot G + \nabla G^T\nabla G \approx \nabla G^T\nabla G$, and $\nabla G = \nabla F|_{x_V=x_V^{(k)}}$,\\
$\nabla^2 g \approx F|_{x_V=x_V^{(k)}}^T F|_{x_V=x_V^{(k)}} \rightarrow\\$
$\nabla^2 g \approx \left( \sum\limits_{j=1}^{m-1} \frac{\partial F_j}{\partial x_{V_u}}|_{x_{V_u}=x_{V_u}^{(k)}} \frac{\partial F_j}{\partial x_{V_v}}|_{x_{V_v}=x_{V_v}^{(k)}} \right)_{u,v=1,2,3}$\\
where $(x_{V_1},x_{V_2},x_{V_3})=(x_{V},y_{V},z_{V})$ and $(x_{V_1}^{(k)},x_{V_2}^{(k)},x_{V_3}^{(k)})=(x_{V}^{(k)},y_{V}^{(k)},z_{V}^{(k)})$.\\
Finally, plug the approximate of the Newton's Step, $S^{(k)} \approx S^{(k,l)}$, into $x^{(k+1)} = x^{(k)} + S^{(k)}$ to update Newton's Method.
}

\item[Exercise 15.] Think about the number of solutions obtained by analyzing four satellite signals with an unknown vehicle time $t_V$. This is an open ended question that will not be graded!

\blue{With 4 satellite signals we end up with 2 solutions after triangulation. Luckily we can eliminate one of the solutions as it usually won't make sense in the context. It may be a point out in space or within the earth.
}

\item[Exercise 16.] I gave an early draft of this assignment to my friend Meg Ikkal Anna Liszt. After muttering about the federal deficit she said that she has been talking to the Air Force (who operate GPS) for years. She does not understand why they are being so hard on themselves. She could save them billions of dollars because to determine position and altitude you only need three satellites, not four! Three satellites would give you three components of position, once you know position you can compute true run time to the satellite, and from that you can compute the current time. She thinks that the Air Force is not implementing this approach because they don't want to pay her fee of 10\% of the savings in launch costs of satellites alone. What do you think of this?

\blue{
The possible position of the vehicle as measured by a single satellite is represented by a sphere around that satellite in three dimensions. With only three satellites as Meg suggests, the intersection of the three satellite's spheres is two distinct points. As in (15), it may be possible to cancel the wrong solution but it could prove to be problematic when real world uncertainties are introduced. With four satellites, however, this problem is no longer present since the fourth satellite will intersect the correct position or near the correct position when real world uncertainties are taken into account and yield a unique solution with relatively small uncertainties.
}


\item[Exercise 17.] After venting her frustration about the federal deficit Meg went to task with \textit{me}. She said that ``you academic types'' like to be so cumbersome. She thinks we don't use ``common sense'' because the very phrase isn't rooted in Latin or Greek. Why, she says, do I have to have integers \textbf{NS} and \textbf{EW} to indicate which hemisphere I'm on? Why, she says, don't I just make the degrees positive or negative? Indeed, why not?

\blue{
It fixes some problems when converting between coordinate systems. For example, when converting a point $(x,y)$ on the cartesian plane to the angle $\lambda$, the function $atan$ is used. However, $atan(\frac{y}{x}) = atan(\frac{-y}{-x})$ even though $(x,y)$ and $(-x,-y)$ are clearly different points. This error isn't fixed by any of the other variables that are transformed since none of them use $x$ and $y$ explicitly.
}

\end{enumerate}

\end{document}  
