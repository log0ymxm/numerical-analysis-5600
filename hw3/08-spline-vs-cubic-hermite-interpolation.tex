\textbf{Spline versus Cubic Hermite Interpolation.} Let the function
$s(x)$ be defined by
\[
s(x) = \left\{
\begin{aligned}
(\gamma - 1)(x^3 - x^2) + x + 1 &\,\,\,\, \text{if}\,\, x\in [0,1] \\
\gamma x^3 - 5 \gamma x^2 + 8 \gamma x - 4 \gamma + 2 &\,\,\,\, \text{if}\,\, x\in [1,2]
\end{aligned}
\right.
\]

\begin{enumerate}
\item Show that $s$ is piecewise cubic Hermite interpolant to the
  data:
\[
s(0)=1, \;\;\; s(1)=s(2)=2, \;\;\; s^\prime(0)=1, \;\;\; s^\prime(1) =
\gamma, \;\;\; s^\prime(2)=0
\]
\item For what value of $\gamma$ does $s$ become a cubic spline?
\end{enumerate}

{\color{blue}

\[
\begin{aligned}
s^\prime(x) = \left\{
\begin{aligned}
(\gamma - 1)(3 x^2 - 2x) + 1 &\,\,\,\, \text{if}\,\, x\in [0,1] \\
3 \gamma x^2 - 10 \gamma x + 8 \gamma &\,\,\,\, \text{if}\,\, x\in [1,2]
\end{aligned}
\right.
\end{aligned}
\]

\begin{enumerate}
\item
\[
\begin{aligned}
s(0) &= 1 \\
s(1) &= s(2) = 2 \\
s^\prime(0) &= 1 \\
s^\prime(1) &= \gamma \\
s^\prime(2) &= 0 \\
% TODO
\end{aligned}
\]

\item

\begin{align*}
% s''(x)
s^{\prime\prime}(x) = \left\{
\begin{aligned}
(\gamma - 1)(6x - 2)
&\,\,\,\, \text{if}\,\, x\in [0,1] \\
6 \gamma x - 10 \gamma
&\,\,\,\, \text{if}\,\, x\in [1,2]
\end{aligned}
\right.
\end{align*}

% TODO the basic difference between the two is that a piece-wise cubic
% Hermite is C^1 at the x_i and interpolates the function values and
% it's derivatives at the points while a cubic spline is C^2 at the
% x_i without interpolating the function's derivatives at the points.
% So we only need to choose a gamma s.t. the interpolation is C^2 at
% 1. Also, we may need to make sure the spline's second derivative is
% 0 at both the endpoints to satisfy the 'natural' spline conditions.c4


\end{enumerate}

}
