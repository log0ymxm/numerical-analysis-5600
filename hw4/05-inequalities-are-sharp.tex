\textbf{(Inequalities are sharp.)} Explain the meaning of \[
\frac{1} {\|A\| \|A^{-1}\|}
\frac{\|r\|} {\|b\|}
\le
\frac{\|e\|}{\|x\|}
\le
\|A\| \|A^{-1}\|
\frac{\|r\|} {\|b\|}
\] and show how we derived these inequalities in class.

\begin{itemize}
\item For a general matrix $A$, and $\| \cdot \| = \| \cdot \|_2$,
  show that there are non-trivial examples (i.e. $x \ne 0 \ne e$)
  where the right hand inequality is satisfied with equality in (3).
  Do the same for the left hand inequality in (3).
\item Repeat part a. for $\| \cdot \| = \| \cdot \|_\infty$.
\end{itemize}

{\color{blue}

$\frac{\| e \|}{\| x \|}$ is the relative error of our linear system,
which may not be computable.
$\frac{\| r \|}{\| b \|}$ is a computable value that we can combine
with the condition number $\| A \| \| A^{-1} \|$ to estimate what the
relative error of our system is.

We say that our system is ill-conditioned if it has a large value for
the condition number, which would mean that the above inequality
doesn't confine the relative error very well. It's well-conditioned
system if the condition number is low, and we can see that the
inequality will give us a narrow bounds and a better estimation of the
relative error.

To derive these equations we have the following equations from our
linear system,

\begin{align*}
Ax &= b \\
A^{-1}b &= x \\
Ae &= r \\
A^{-1}r &= e
\end{align*}

Where $e$ is our error $x - \hat{x}$ therefore $r$ is a relation of
the error and our system.

Using the Cauchy-Schwarz inequality we know that the norms of the
above equations follow these inequalities,
\begin{align*}
\| b \| &\le \| A \| \| x \| \\
\| x \| &\le \| A^{-1} \| \| b \| \\
\| r \| &\le \| A \| \| e \| \\
\| e \| &\le \| A^{-1} \| \| r \| \\
\end{align*}

We can divide the first inequality from the fourth and the second from
the third (we take the reciprocal reversing the inequality and multiply) to arrive at
\begin{align*}
\frac{\| e \|}{\| A \| \|x \|} &\le \frac{\| A^{-1} \| \| r \|}{\| b \|} \\
\frac{\| r \|}{\| A^{-1} \| \| b \|} &\le \frac{\| A \| \| e \|}{\| x \|} \\
\end{align*}

Now with the new set of inequalities we multiply the first by $\|
A \|$ and the second by $\frac{1}{\| A\|}$ to get to the original inequality,
\begin{align*}
\frac{1} {\|A\| \|A^{-1}\|} \frac{\|r\|} {\|b\|}
\le
\frac{\|A\|\|e\|}{\| A\|\|x\|}
\le
\|A\| \|A^{-1}\| \frac{\|r\|} {\|b\|} \\
\Rightarrow \frac{1} {\|A\| \|A^{-1}\|} \frac{\|r\|} {\|b\|}
\le
\frac{\|e\|}{\|x\|}
\le
\|A\| \|A^{-1}\| \frac{\|r\|} {\|b\|}
\end{align*}

For $\| \cdot \| = \| \cdot \|_2$ we have that
$\| A \| = \max_{Ax = \lambda x} \{\lambda\} = \lambda_1$ or
the square root of the largest eigenvalue, thus
the norm of $A^{-1}$ will be one over the square root of the smallest eigenvalue
$\| A^{-1} \| = \frac{1}{\min_{Ax = \lambda x} \{\lambda\}}
= \frac{1}{\lambda_n}$ since the eigenvalues of an
inverted matrix are reciprocals. Now our condition number is
$ \|A\| \|A^{-1}\| = \frac{\lambda_1}{\lambda_n}$, and our
relative error is bounded by,

\begin{align*}
\frac{\lambda_n}{\lambda_1}
 \frac{\|r\|} {\|b\|}
\le
\frac{\|e\|}{\|x\|}
\le
\frac{\lambda_1}{\lambda_n}
 \frac{\|r\|} {\|b\|}
\end{align*}

Here inequality is achieved when $\lambda_1 = \lambda_n$. We can find
this case trivially with the identity matrix, i.e.
$A = \begin{bmatrix} 1 & 0 \\ 0 & 1 \end{bmatrix}$.

With $\| \cdot \| = \| \cdot \|_{\infty}$ our matrix norm is
$\| A \| = \max\{ \|a_{ij}\| \} = a_{\max} $ and
$\| A^{-1} \| = \frac{1}{ \min\{\|a_{ij}\|\} } = \frac{1}{a_{\min}}$. This means that our
condition number is
$ \|A\| \|A^{-1}\| = \frac{a_{\max}}{a_{\min}} $, and our inequality will
have the case of equality when $a_{\max} = a_{\min}$. An example of this
is $A = \begin{bmatrix}1 & 0 \\ 0 & -1\end{bmatrix}$.

}
