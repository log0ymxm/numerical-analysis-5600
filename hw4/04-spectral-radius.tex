\textbf{(Spectral Radius.)} We saw that the spectral radius of a
(square) matrix $A$ never exceeds an induced matrix norm of $||A||$.
It can be shown that for any particular matrix $A$ one can find a
vector norm such that the induced matrix norm of $A$ is arbitrarily
close to the spectral radius of $A$. Does the spectral radius itself
define a norm? Why, or why not?



{\color{blue}
\begin{align}
Consider\,  0&=\begin{pmatrix}
0 & 0\\ 
0 & 0
\end{pmatrix}
and\, A=
\begin{pmatrix}
0 & 0\\ 
0 & 0
\end{pmatrix}\\
\end{align}

Clearly, A$\neq0\\$
The eigenvalues of 0 and A are$ \{0,0\}\\$
$\Rightarrow \rho(A)=\rho(0)=max \{0,0\}\\$
By definition, a norm requires the property\\
$||A|| =0 \to A=0$\\
Here, we have $\rho(A)=0, but A\neq 0$\\
Therefore, $\rho$ is \underline{not} a matrix norm.

}
